%-------------------------
% Резюме в LaTeX
% Автор : Глазов Михаил
% Основано на: https://github.com/sb2nov/resume
% Лицензия : MIT
%------------------------

\documentclass[letterpaper,11pt]{article}

\usepackage[utf8]{inputenc}
\usepackage[T2A]{fontenc}
\usepackage[russian]{babel}
\usepackage{latexsym}
\usepackage[empty]{fullpage}
\usepackage{titlesec}
\usepackage{marvosym}
\usepackage[usenames,dvipsnames]{color}
\usepackage{verbatim}
\usepackage{enumitem}
\usepackage[hidelinks]{hyperref}
\usepackage{fancyhdr}
\usepackage{tabularx}
\input{glyphtounicode}

\pagestyle{fancy}
\fancyhf{} % очищаем все поля
\fancyfoot{}
\renewcommand{\headrulewidth}{0pt}
\renewcommand{\footrulewidth}{0pt}

% Настройка полей
\addtolength{\oddsidemargin}{-0.5in}
\addtolength{\evensidemargin}{-0.5in}
\addtolength{\textwidth}{1in}
\addtolength{\topmargin}{-.5in}
\addtolength{\textheight}{1.0in}

\urlstyle{same}

\raggedbottom
\raggedright
\setlength{\tabcolsep}{0in}

% Форматирование разделов
\titleformat{\section}{
  \vspace{-4pt}\scshape\raggedright\large
}{}{0em}{}[\color{black}\titlerule \vspace{-5pt}]

\pdfgentounicode=1

% Пользовательские команды
\newcommand{\resumeItem}[1]{
  \item\small{
    {#1 \vspace{-2pt}}
  }
}

\newcommand{\resumeSubheading}[4]{
  \vspace{-2pt}\item
    \begin{tabular*}{0.97\textwidth}[t]{l@{\extracolsep{\fill}}r}
      \textbf{#1} & #2 \\
      \textit{\small#3} & \textit{\small #4} \\
    \end{tabular*}\vspace{-7pt}
}

\newcommand{\resumeSubSubheading}[2]{
    \item
    \begin{tabular*}{0.97\textwidth}{l@{\extracolsep{\fill}}r}
      \textit{\small#1} & \textit{\small #2} \\
    \end{tabular*}\vspace{-7pt}
}

\newcommand{\resumeProjectHeading}[2]{
    \item
    \begin{tabular*}{0.97\textwidth}{l@{\extracolsep{\fill}}r}
      \small#1 & #2 \\
    \end{tabular*}\vspace{-7pt}
}

\newcommand{\resumeSubItem}[1]{\resumeItem{#1}\vspace{-4pt}}

\renewcommand\labelitemii{$\vcenter{\hbox{\tiny$\bullet$}}$}

\newcommand{\resumeSubHeadingListStart}{\begin{itemize}[leftmargin=0.15in, label={}]}
\newcommand{\resumeSubHeadingListEnd}{\end{itemize}}
\newcommand{\resumeItemListStart}{\begin{itemize}}
\newcommand{\resumeItemListEnd}{\end{itemize}\vspace{-5pt}}

\begin{document}

\begin{center}
    \textbf{\Huge \scshape Глазов Михаил Николаевич} \\ \vspace{1pt}
    \small +7 (961) 339-71-42 $|$ \href{mailto:mihail.glazov2015@yandex.ru}{\underline{mihail.glazov2015@yandex.ru}} $|$ 
    \href{https://t.me/mikhailglazov}{\underline{Telegram: @mikhailglazov}} $|$
    \href{https://github.com/skewbekkebweks}{\underline{github.com/skewbekkebweks}} $|$
    \href{https://codeforces.com/profile/skewbekkebweks}{\underline{codeforces.com/profile/skewbekkebweks}}
\end{center}

%-----------ОБРАЗОВАНИЕ-----------
\section{Образование}
  \resumeSubHeadingListStart
    \resumeSubheading
      {Национальный исследовательский университет «Высшая школа экономики»}{Москва}
      {Факультет компьютерных наук, Прикладная математика и информатика}{2028}
  \resumeSubHeadingListEnd

%-----------ОПЫТ РАБОТЫ-----------
\section{Опыт работы}
  \resumeSubHeadingListStart
    \resumeSubheading
      {Yandex}{Москва}
      {Стажёр-разработчик}{Июль 2025 -- Октябрь 2025}
      \resumeItemListStart
        \resumeItem{\textbf{Разрабатывал и внедрял критически важные компоненты для высоконагруженных антифрод-сервисов.} В частности, спроектировал и реализовал механизм троттлинга для защиты системы от перегрузок и DDoS-атак}
        \resumeItem{\textbf{Проводил оптимизацию и рефакторинг существующего C++ кода.} Улучшил производительность модуля расчета статистики и вывел из эксплуатации устаревшие компоненты, что привело к упрощению кодовой базы и повышению ее поддерживаемости}
        \resumeItem{\textbf{Развивал систему мониторинга и observability.} Добавил метрики для отслеживания ключевых бизнес-показателей (например, доля успешно пройденных captcha) и настроил алертинг для контроля работоспособности инфраструктуры (доступность подов), что позволило быстрее реагировать на инциденты}
        \resumeItem{\textbf{Участвовал в поддержке инфраструктуры для работы с ML-моделями.} Отвечал за корректную интеграцию и поставку актуальных версий моделей в релизы антифрод-системы}
        \resumeItem{\textbf{Улучшал инструменты для анализа и отладки.} Упростил интерфейсы для работы с историческими данными и расширил возможности для диагностики производительности сервисов на уровне отдельных хостов}
      \resumeItemListEnd

    \resumeSubheading
      {Хакатон Finodays}{Москва}
      {Python-разработчик}{Август 2023 -- Ноябрь 2023}
      \resumeItemListStart
        \resumeItem{Разработал backend для финтех-проекта на FastAPI с использованием Docker}
        \resumeItem{Команда стала победителем на Finopolis 2024}
      \resumeItemListEnd

    \resumeSubheading
      {Хакатон "Лидеры цифровой трансформации"}{Москва}
      {Python-разработчик}{Май 2024 -- Июнь 2024}
      \resumeItemListStart
        \resumeItem{Разработал алгоритм распределения заявок маломобильных пассажиров по волонтёрам}
        \resumeItem{Реализовал REST API с использованием FastAPI}
        \resumeItem{Команда стала финалистом хакатона}
      \resumeItemListEnd
    
    \resumeSubheading
      {NLogN}{Москва}
      {Преподаватель алгоритмов}{Июнь 2024 -- Июль 2024}
      \resumeItemListStart
        \resumeItem{Чтение лекций и работа с детьми}
      \resumeItemListEnd
  \resumeSubHeadingListEnd

%-----------ПРОЕКТЫ-----------
\section{Проекты}
    \resumeSubHeadingListStart

      \resumeProjectHeading
          {\textbf{Маркетплейс ЦФА} $|$ \emph{Python, FastAPI, Docker} | \href{https://github.com/avalanche05/finodays}{\underline{github}}}{2023}
          \resumeItemListStart
            \resumeItem{Разработан backend для маркетплейса финансовых активов}
            \resumeItem{Реализовано REST API с использованием Docker}
          \resumeItemListEnd
        
      \resumeProjectHeading
          {\textbf{Сервис мониторинга заявок маломобильных пассажиров} $|$ \emph{Python, FastAPI, Алгоритмы} | \href{https://github.com/nizhgo/lct-2024}{\underline{github}}}{2024}
          \resumeItemListStart
            \resumeItem{Сервис для мониторинга и адаптивного распределения заявок}
            \resumeItem{Реализован эффективный алгоритм сопоставления пассажиров и волонтёров}
          \resumeItemListEnd
          
      \resumeProjectHeading
          {\textbf{Архиватор файлов} $|$ \emph{C++, Алгоритм Хаффмана} | \href{https://github.com/skewbekkebweks/archiver/tree/main/tasks/archiver}{\underline{github}}}{2024}
          \resumeItemListStart
            \resumeItem{Реализовано сжатие файлов с использованием алгоритма Хаффмана}
          \resumeItemListEnd
          
      \resumeProjectHeading
          {\textbf{Приложение для занятий спортом} $|$ \emph{Java, Android} | \href{https://github.com/skewbekkebweks/chick}{\underline{Android}} | \href{https://github.com/skewbekkebweks/chick_server}{\underline{backend}}}{2023}
          \resumeItemListStart
            \resumeItem{Разработано мобильное приложение для фитнес-трекинга с серверной частью}
          \resumeItemListEnd
      \resumeProjectHeading
          {\textbf{Решение задачи SVHN (computer vision)} $|$ \emph{ML, Python} | \href{https://github.com/GrishaTS/Ya-Intensive-ML}{\underline{github}}}{2023}
          \resumeItemListStart
            \resumeItem{Обученены собственные модели для решения задачи}
          \resumeItemListEnd
    \resumeSubHeadingListEnd
  

%-----------ДОСТИЖЕНИЯ-----------
\section{Олимпиадный опыт}
  \resumeSubHeadingListStart
    \resumeItem{Участник финала ВКОШП (Всероссийская командная олимпиада школьников по программированию)}
    \resumeItem{Участник заключительного этапа ВсОШ (Всероссийская олимпиада школьников) по информатике}
    \resumeItem{Победитель МОШ (Московская олимпиада школьников) по информатике}
    \resumeItem{Призёр Высшей пробы по информатике}
    \resumeItem{Участник 1/4 финала ICPC (International Collegiate Programming Contest)}
  \resumeSubHeadingListEnd

%-----------КУРСЫ-----------
\section{Курсы}
  \resumeSubHeadingListStart
    \resumeItem{Интенсив по ML (Академия Яндекса)}
    \resumeItem{Выпускник Яндекс.Лицея (программирование на Python)}
    \resumeItem{Выпускник IT Школы Samsung (разработка на Java (Android и Spring-boot))}
    \resumeItem{T-Поколение, курс "Алгоритмы и структуры данных"}
  \resumeSubHeadingListEnd

%-----------НАВЫКИ-----------
\section{Технические навыки}
 \begin{itemize}[leftmargin=0.15in, label={}]
    \small{\item{
     \textbf{Языки программирования}{: Python, C++, Java, SQL (PostgreSQL)} \\
     \textbf{Фреймворки}{: FastAPI, Django, Android SDK, Spring Boot} \\
     \textbf{Инструменты}{: Git, Docker, MongoDB, REST API} \\
     \textbf{Алгоритмы}{: Структуры данных, Олимпиадное программирование} \\
     \textbf{Языки}{: Русский (родной), Английский (B2)}
    }}
 \end{itemize}


\end{document}